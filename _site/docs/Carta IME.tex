\documentclass[12pt,a4paper]{article}
\usepackage[utf8]{inputenc}
\usepackage[T1]{fontenc}
\usepackage[brazilian]{babel}
\usepackage{lipsum}
\usepackage{fancyhdr}
\usepackage{color}

%links
\usepackage{hyperref}
\hypersetup{
	colorlinks=true,
	linkcolor=black,
	citecolor=black,
	filecolor=black,
	urlcolor=blue
}

%cabeçalho
\fancyhf{} % limpa os cabecalhos e rodapés
\fancyhead[L]{ARTHUR BOARI} % define o cabeçalho personalizado
\fancyhead[R]{\thepage / \pageref{LastPage}}
\pagestyle{fancy} % sem definir esse comando, o cabeçalho personalizado não é exibido

\hyphenation{hardware software Li-nux am-bien-te diag-nos-ti-car coor-de-na-ção 
	FAE-PE Recovery TelEduc Williams UFLA }

\begin{document}
	
	\begin{flushright}
		\href{http://lattes.cnpq.br/7113151378838886}{Arthur Boari}\\
		Rua Ametista, 317 – Parque das Pedras Preciosas\\
		Lavras, Minas Gerais\\
		\href{mailto:eng.arthurboari@gmail.com}{eng.arthurboari@gmail.com}
	\end{flushright}


	\begin{flushleft}
		À Comissão de Seleção de Candidatos para Doutorado em Estatística\\
		Departamento de Estatística\\
		Instituto de Matemática e Estatística\\
		Universidade de São Paulo\\
		São Paulo, São Paulo
	\end{flushleft}

	Prezados membros da Comissão, \\
	
	%Inicio agradecendo a oportunidade de ter a minha candidatura apreciada no presente edital. Sou bacharel em Engenharia Ambiental e Sanitária e mestrando em Engenharia Ambiental pela Universidade Federal de Lavras (UFLA). A motivação em realizar o doutoramento em estatística vem do meu anseio em expandir o meu conhecimento em estatística, visto que a minha área de pesquisa, ciências atmosféricas, é bastante dependente de análises estatísticas robustas.
	
	%\section*{Graduação}
	
	%Em 2014, iniciei a minha graduação em Engenharia Ambiental e Sanitária sob a grade \href{https://sig.ufla.br/modulos/publico/matrizes_curriculares/index.php}{G019 - 2013/2}, há apenas a disciplina de Estatística Básica (GES101) no 3º módulo do curso. Nela são contemplados tópicos de estatística descritiva, probabilidade e suas distribuições, amostragem, correlação e regressão, entre outros. Após essa disciplina, tópicos da estatística são revistos em apenas duas disciplinas, uma no 6º módulo (GRH103	- Hidrologia I) e outra no 9º módulo (GAM114 -	Modelagem de Processos Ambientais).
	
	%O Trabalho de Conclusão de Curso, conduzido com a orientação da \href{http://lattes.cnpq.br/3943657653311716}{Prof. Dra. Sílvia de Nazaré Monteiro Yanagi}, teve como objeto de estudo a interação de métodos de estimativa de evapotranspiração potencial (Penman-Monteith – padrão FAO, Hargreaves \& Samani, Makkink, e Thornthwaite) na metodologia de balanço hídrico climatológico (BHC – Thornthwaite \& Matter). Foi desenvolvido através de dados meteorológicos obtido através do Instituto Nacional de Meteorologia (INMET), processamento em planilhas eletrônicas e uso de linguagem de programação R. Em termos estatísticos foram usadas estatística descritiva, índices de Willmott (acurácia dos modelos) e de desempenho (Camargos e Sentelhas – acurácia e precisão dos métodos), e regressão linear. O trabalho, que ainda não foi publicado, está anexado na documentação da candidatura. Em minha colação de grau fui agraciado com o 1º lugar no \href{https://ufla.br/noticias/ensino/14064-etapa-concluida-mais-de-500-novos-profissionais-formados-pela-ufla}{Prêmio de Mérito Acadêmico em Engenharia Ambiental - 2020}. 
	
	%\section*{Mestrado}
	%O ingresso ao mestrado ocorreu em novembro de 2020 sob orientação do \href{http://lattes.cnpq.br/5059318976988668}{Prof. Dr. Marcelo Vieira-Filho}. A pesquisa foi voltada a poluição do ar, em especial, a registrada nas capitais da região Sudeste do Brasil. Os dados utilizados foram obtidos através dos portais das entidades governamentais estaduais, sendo processados em linguagem R. Em termo estatístico, foram realizados a descrição dos dados, bem como testes de normalidade (Anderson-Darling) e de tendências (Mann-Kendall, Sen’s Slope e Cox-Stuart). Os resultados parciais foram publicados em congressos a nível local e nacional, enquanto os resultados finais foram submetidos em dois artigos: ..... 
	
	%Em relação as disciplinas, como obrigatória, a PEA507 – Tratamento Estatístico de Dados Ambientais trouxe uma revisão de conceitos estatísticos (descritiva, testes estatísticos, regressão, correlação e modelos lineares, e métodos numéricos) que foram aplicados na elaboração de um artigo. De eletivas, cursei PEX519 – Séries temporais, PEX518 – Regressão, e PEX820 – Data manipulation and vizualization. Em especial, a PEX – ST foi importante para introdução da aplicação de modelos SARIMA
	
	%\section*{Atividades extracurriculares}
	%Em termo de atividades extracurriculares, desenvolvi, por quase quatro anos (03/2016 a 09/2019), pesquisas na área de tratamentos de água e esgoto através de análises laboratoriais de amostras da ETA e ETE do campus sede da UFLA. Durante esse período foram publicados vários resumos simples e expandidos em congressos a nível local e nacional, onde a estatística descritiva foi amplamente utilizada.
	
	%No começo de 2019 alterei a minha área de pesquisa para as ciências atmosféricas e, como catalisador, ingressei no Núcleo de Estudos em Poluição Urbana e Agroindustrial (NEP UAI) – coordenado pelo Prof. Dr. Marcelo Vieira-Filho. A minha contribuição estendeu até 03/2022, quando já estava no mestrado. Em termos de gestão, fui conselheiro de comunicação, geral e de projetos, onde desenvolvi habilidades comunicacionais e de liderança. No campo das pesquisas, forma contempladas a poluição sonora no campus sede da UFLA, o impacto das medidas de lockdown no início da pandemia da COVID-19 (com coautoria de um artigo internacional e participação em congresso internacional), tendência temporal da concentração de poluentes do ar na Região Metropolitana de Belo Horizonte (publicação de resumo expandido em congresso internacional), e anomalias de precipitação e temperatura em Lavras, MG.
	
	Me chamo Arthur Boari, sou natural de Lavras - MG, bacharel em Engenharia Ambiental e Sanitária e mestre em Engenharia Ambiental pela Universidade Federal de Lavras (UFLA), e escrevo para expressar o meu interesse em realizar o meu doutoramento em estatística pela Universidade de São Paulo (USP).
	
	Iniciei a minha graduação em Engenharia Ambiental e Sanitária em 2014 e a concluí em 2020 com o recebimento do \href{https://ufla.br/noticias/ensino/14064-etapa-concluida-mais-de-500-novos-profissionais-formados-pela-ufla}{Prêmio de Mérito Acadêmico em Engenharia Ambiental - 2020}. Acredito que as minhas experiências extracurriculares são um diferencial da minha formação. Do terceiro ao décimo período desenvolvi pesquisa na área de tratamento de água e esgoto, sumariamente trabalhando em análises físico-químicas e biológicas, que renderam várias publicações em congressos \href{https://prp.ufla.br/ciuflasig/library.php}{locais} e nacionais. Nesse período fui bolsista voluntário de iniciação científica e bolsista de aprendizado técnico.
	
	No início de 2019 decidido a alterar a minha área de pesquisa optei por estudar processos atmosféricos, como poluição sonora e do ar. Para auxiliar nessa empreitada, prestei processo seletivo do \href{https://sites.google.com/ufla.br/nepuai?pli=1}{Núcleo de Estudos em Poluição Urbana e Agroindustrial (NEP UAI)} – coordenado pelo \href{http://lattes.cnpq.br/5059318976988668}{Prof. Dr. Marcelo Vieira-Filho}. A minha contribuição estendeu até 03/2022, quando já estava no mestrado. Em termos de gestão, fui conselheiro de comunicação, geral e de projetos, onde desenvolvi habilidades comunicacionais e de liderança. No campo das pesquisas, forma contempladas a poluição sonora no \textit{campus} sede da UFLA (resultou na publicação de um \href{https://sites.google.com/ufla.br/nepuai/publica%C3%A7%C3%B5es/livros?authuser=0}{livro}), o impacto das medidas de lockdown no início da pandemia da COVID-19 (com coautoria de um \href{https://link.springer.com/article/10.1007/s11869-020-00959-8}{artigo internacional} e participação em congresso internacional), tendência temporal da concentração de poluentes do ar na Região Metropolitana de Belo Horizonte (publicação de resumo expandido em  \href{https://www.inicepg.univap.br/cd/INIC_2021/anais/arquivos/RE_0771_0575_01.pdf}{congresso} internacional), e anomalias de precipitação e temperatura em Lavras, MG. Através do NEP UAI conheci a linguagem R de programação, e desde então tenho desenvolvido habilidades que envolvem a produção de mapas e gráficos (em especial, o pacote \textit{ggplot2}), dashboards (\textit{flexdashboard}), produção de documentos em \textit{rmarkdown}, e a construção de um portfólio no \href{https://arthurboari.github.io/arthurboari/}{\textit{GitHub}}.
	
	Ainda na graduação, desenvolvi o meu Trabalho de Conclusão de Curso sob a orientação da \href{http://lattes.cnpq.br/3943657653311716}{Prof. Dra. Sílvia de Nazaré Monteiro Yanagi}, onde estudei a interação de métodos de estimativa de evapotranspiração potencial (Penman-Monteith – padrão FAO, Hargreaves \& Samani, Makkink, e Thornthwaite) na metodologia de balanço hídrico climatológico (BHC, metodologia de Thornthwaite \& Matter). Foi desenvolvido através de dados meteorológicos obtido através do Instituto Nacional de Meteorologia (INMET), processamento em planilhas eletrônicas e uso de linguagem de programação R. Em termos estatísticos foram usadas estatística descritiva, índices de Willmott (acurácia dos modelos) e de desempenho (Camargos e Sentelhas – acurácia e precisão dos métodos), e regressão linear. O trabalho, que ainda não foi publicado, está anexado na documentação da candidatura.
	
	A pós-graduação, nível mestrado, em Engenharia Ambiental começou em novembro de 2020 sob orientação do \href{http://lattes.cnpq.br/5059318976988668}{Prof. Dr. Marcelo Vieira-Filho}. A pesquisa foi voltada a poluição do ar, em especial, a registrada nas capitais da região Sudeste do Brasil. Os dados utilizados foram obtidos através dos portais das entidades governamentais estaduais, sendo processados em linguagem R. A motivação desse estudo foi verificar a tendência da concentração de material particulado (MP$_{2.5}$ e MP$_{10}$) e ozônio (O$_3$) nessas capitais, e, para isso, utilizei testes de normalidade (Anderson-Darling) e de tendências (Mann-Kendall, Sen’s Slope e Cox-Stuart). Os resultados parciais foram publicados em congressos a nível local e \href{http://www.meioambientepocos.com.br/ANAIS2022/76%20-%20244016_crescimento-da-concentrao-de-materiais-particulados-e-oznio-em-capitais-brasileiras.pdf}{nacional}, enquanto os resultados finais foram submetidos em \colorbox{red}{dois} artigos: ..... Em destaque, verifiquei tendências de aumento da concentração de MP$_{2.5}$ e O$_3$ para São Paulo, além de tendência de aumento das ultrapassagens dos padrões internacionais.
	
	Em relação as disciplinas, como obrigatória, a PEA507 – Tratamento Estatístico de Dados Ambientais trouxe uma revisão de conceitos estatísticos (descritiva, testes estatísticos, regressão, correlação e modelos lineares, e métodos numéricos) que foram aplicados na elaboração de um artigo. De eletivas, cursei PEX519 – Séries temporais (modelos ARMA, ARIMA, SARIMA, GARCH, dentre outros), PEX518 – Regressão (regressão linear simples e multivariada), e PEX820 – Data manipulation and vizualization. A seleção dessas disciplinas originaram da necessidade de aprofundamento e orientação em relação ao projeto do mestrado. O grande volume de equações complexas da PEX519 me motivou a aprender a escrever documentos em \LaTeX, e desde então tenho aperfeiçoado essa habilidade.
	
	A minha motivação em pleitear uma vaga no processo seletivo do doutorado em estatística vem
	
	\label{LastPage}
\end{document}